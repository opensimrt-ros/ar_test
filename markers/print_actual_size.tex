%
% @author      : frekle (frekle@bml01.mech.kth.se)
% @created     : 17/05/2023
% @filename    : print_actual_size.tex
%

\documentclass{article}

\usepackage[utf8]{inputenc}
\usepackage[a4paper, top=10mm, bottom=10mm, left=15mm, right=15mm]{geometry}
\usepackage[all]{background}
\usepackage{datetime}
%\usepackage{url}
\usepackage{hyperref}

\SetBgContents{
    \begin{tabular}{c c}
        AR fiducial markers with actual size.\\

    \end{tabular}

%\url{http://tex.stackexchange.com/}
}% Set contents
\backgroundsetup{color=black}

\SetBgPosition{current page.west}% Select location
\SetBgVshift{-1.0cm}% Add vertical shift (results in a shift in x direction due to rotation)
\SetBgOpacity{1.0}% Select opacity
\SetBgAngle{90.0}% Select rotation of logo
\SetBgScale{1.0}% Select scale factor of logo

\newcommand{\bb}[1]{\textbf{\displaydate{#1}}}

\renewcommand{\baselinestretch}{1.2}
\makeatletter         
\renewcommand\maketitle{
{\raggedright % Note the extra {
\begin{center}
{\LARGE
%\bfseries 
\sffamily 
\@title }\\[2ex] 
{
%\Large
\@author}\\[2ex] 
%\@prof \\
\@date\\[6ex]
\end{center}
}}
\makeatother

\usepackage[
backend=biber,
style=alphabetic,
sorting=ynt
]{biblatex}
\addbibresource{bib.bib}

\title{\textbf{Upperbody single arm AR markers example }
}
\author{ Frederico B. Klein }
\date{\today}
\usepackage{graphicx}
\newenvironment{Marker}[1]
    {\begin{center}
	    \vspace*{2cm}
	    \begin{tabular*}{\textwidth}{@{\extracolsep{\fill} } l r}
	    \rotatebox{90}{#1} &
    }
    { 
    \end{tabular*} 
	    \vspace*{2cm}
    \end{center}
    }


\begin{document}
\maketitle

\thispagestyle{empty}

We are using latex to get accurate scaling here. There are easier ways of doing this, but since you are printing a whole page anyway, might as well include some instructions. 

This will print the simplest version of the upperbody AR markers setup. A more complete version would have the complete bundles for each of the bodies. Same can be done for lowerbody. Note the makers need to occupy a large portion of the camera canvas or you will be losing angulare resolution. 

Markers are generated using the script from ar\_track\_alvar with:

\texttt{rosrun ar\_track\_alvar CreateMarker}
%For more information see:\cite{}

%TORAX:
\begin{Marker}{TORAX}
    \centering
    \includegraphics[width=13cm]{MarkerData_0}
\end{Marker}

%\includegraphics[width=10cm]{MarkerData_0.png}
%\includegraphics[width=10cm]{MarkerData_1.png}
%\pagebreak

%UPPERARM:\\
\begin{Marker}{UPPERARM}
    \centering
    \includegraphics[width=13cm]{MarkerData_2}
\end{Marker}
%\includegraphics[width=10cm]{MarkerData_3.png}
%\includegraphics[width=10cm]{MarkerData_4.png}
%\includegraphics[width=10cm]{MarkerData_5.png}

%FOREARM:\\
\begin{Marker}{FOREARM}
    \centering
    \includegraphics[width=9cm]{MarkerData_6}
\end{Marker}
%\includegraphics[width=10cm]{MarkerData_7.png}
%\includegraphics[width=10cm]{MarkerData_8.png}
%\includegraphics[width=10cm]{MarkerData_9.png}

\printbibliography
\end{document}

